\section{Results}

\subsection{Behavioral analysis}

We performed statistical analysis using both Python and R (The original paper 
use R package to fit the Logistic models). We use the library 
\emph{scikit-learn} in Python and the \emph{glm} function in \emph{stats} 
in R to fit the models. Models from two library yields the same results. 
Following is the box plot of the behavioral loss aversion $\lambda$ 
(median=1.94, mean=2.18, min=0.99, max=0.75). This result is consistent with 
that of the paper, which indicate that participants are indifferent to gambles 
whose gain are appoxiamtely twice as the loss. 
\begin{figure}[H]
\caption{Box plot of the behavioral loss aversion $\lambda$ }
    \centering
        \includegraphics[scale=0.35]{figures/lambda_boxplot.png}
\end{figure}
The accuracy of Logistic models (for 16 participants, there are 16 models in 
total) on the training set yielded a median of 89.78\% (min=80.97\%, 
max=99.21\%). We also did the model evaluation using 10-fold cross-validation, 
they are still performing accuracies of a median of 89.86\% (min=79.92\%, 
max=98.45\%).


\subsection{Linear Regression on BOLD data}

The topic we are interested in exploring is whether loss aversion reflects 
the engagement of distinct emotional processes when potential gains and 
losses are considered. In the process, we want to explore the correlation 
between neural and behavioral loss aversion in whole brain analysis. We also 
want to try to identify the regions of brain that is more activated by this 
loss aversion activity.

Since we want to explore the correlation between neural and behavioral loss 
aversion, the second step is to find out the neural loss aversion. In order 
to find the neural loss aversion, we perform a linear regression on the BOLD 
data against the parametric gain values and the parametric loss values, as 
explained in our model section. (While implementing the linear regression for 
the raw data, we also added linear and quadratic drift in our model. These 
drift terms are modeling for gradual drifts across the time series.)

We are especially interested in the beta coefficients of our parametric gain 
and parametric loss regressors, which are the first two columns in our design 
matrix. To find out the regions with significant positive or negative 
correlation with increasing gain or loss levels, we calculate the t statistics 
for each voxel. Plotting heat maps of the t statistics will show us regions 
with significant parametric increase in fMRI signal to increasing potential 
gains and regions with significant parametric decrease to increasing potential 
losses.

By looking at the heatmaps, we can get a general idea of how potential gains 
and potential losses affect brain activation. We can also identify the areas 
that have large coefficients; these are the areas that the brain activation is 
highly connected to the potential gains and losses. We choose the preprocessed 
data for subject 2 to plot heat maps since the preprocessed data is mapped 
onto the standard brain. We plot slices 31 to 60 from the third dimension of 
the brain (top to bottom). The red color is associated with positive t values 
and blue color is associated with negative t values. We can see that for the 
significant t values for the gain coefficients are mostly positive while the
the significant t values for the loss coefficients are mostly negative.

\begin{figure}[H]
    \centering
        \includegraphics[scale=0.42]{figures/t_gain_standard_sub2.png}
    \caption{t values of the gain coefficients for subject 2}
\end{figure}

\begin{figure}[H]
    \centering
        \includegraphics[scale=0.42]{figures/t_loss_standard_sub2.png}
    \caption{t values of the loss coefficients for subject 2}
\end{figure}

We can see that signigicant areas for the gain coefficients and those of the 
loss coefficients are mostly the same. This suggests that opposite of what 
most people believes that increasing potential losses should affect the areas 
of the brain that mediate negative emotions in decision-making, potential 
losses were represented by decreasing activity in the same areas that are 
sensitive to potential gains.

From the gain and loss coefficients, we can also compute the neural loss
aversion. This serves the next step of looking at the correlation between
neural and behavioral loss aversion. The neural gain and loss coefficients
were broadly distributed and spanned zero, so it is not possible to compute
the ratio of loss to gain coefficients, nor does it make much sense.
Therefore, we compute the neural loss aversion at every voxel by subtracting
the slope of the gain response from the (negative) slope of the loss response.
With the neural loss aversion values calculated, we can explore how loss
aversion affects brain activation when potential gains and losses are
considered.


\subsection{Mixed-effects model on fMRI data}
First, we did the ANOVA test for each subject each voxels, grouping by runs. 
The high proportion of significant ANOVA F-test (after  Bonferroni correction 
under 0.05 significant level) shows that mixed effects model may perform well 
when collapsing three runs into one model. 
\begin{figure}[H]
\caption{Box plot of the proportion of significant ANOVA test}
    \centering
        \includegraphics[scale=0.45]{figures/anova_prop.png}
\end{figure}
Following are the slices of coefficient of gain for subject 002. The 
mixed-effects model for each subject yielded a a median of 9.4\% (min=6.4\%, 
max=21.5\%) and 8.3\% (min=4.6\%, max=15.4\%) of proportion of significant 
coefficient for gain and loss separately. 
\begin{figure}[H]
\caption{heatmap of coefficient of gain for subject 002}
    \centering
        \includegraphics[scale=0.35]{figures/sub002_lme_beta_gain.png}
\end{figure}


