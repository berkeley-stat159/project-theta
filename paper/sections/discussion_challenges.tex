% Subsection of Discussion
\subsection{Discussion of Challenges}

\par \indent One of the major challenges is trying to make this project as 
reproducible as possible while following guidlines on documentation, testing 
functions, and attempting to produce the results of the paper using our limited
understanding of fMRI data. Travis CI bugs with various versions of python, 
coverage failures, and errors with directory/path locations often hinder the 
process of smooth workflows. Collaboration between five group members is no 
doubt difficult as we found it hard to come up with an attainable final goal 
that is still rewarding. 
\par Technically, most of us are new to python programming and reseach using 
git workflows, thus we have only a preliminary understanding of the various 
python resources available for our use. Additionally, lack of statistical 
understanding of some aspects of the paper has urged us to do independent 
research. Yet the disconnent between theory and implementation has been a major 
obstacle because as we try to put our knowledge into practice, we realize that
many pre-packaged software used in the original paper are unavailable to us. 
In addressing this, we have made our best attempt at creating a fMRI analysis 
pipeline. 
\par Some problems can be solved or alleviated by defining checkpoints and making
the effort to re-read the paper and ask questions. Further, as we familiarize 
ourselves more and more  with various python modules and toolkits, results can 
be easier to attain and interpret. Nonetheless, we believe we have made 
significant gain in our understanding of not only fMRI research, but more 
importantly, the process of creating collaborative and reproducible research and
navigating the rough learning road of scientific programming. 
